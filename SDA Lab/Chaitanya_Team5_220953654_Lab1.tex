% Options for packages loaded elsewhere
\PassOptionsToPackage{unicode}{hyperref}
\PassOptionsToPackage{hyphens}{url}
%
\documentclass[
]{article}
\usepackage{amsmath,amssymb}
\usepackage{iftex}
\ifPDFTeX
  \usepackage[T1]{fontenc}
  \usepackage[utf8]{inputenc}
  \usepackage{textcomp} % provide euro and other symbols
\else % if luatex or xetex
  \usepackage{unicode-math} % this also loads fontspec
  \defaultfontfeatures{Scale=MatchLowercase}
  \defaultfontfeatures[\rmfamily]{Ligatures=TeX,Scale=1}
\fi
\usepackage{lmodern}
\ifPDFTeX\else
  % xetex/luatex font selection
\fi
% Use upquote if available, for straight quotes in verbatim environments
\IfFileExists{upquote.sty}{\usepackage{upquote}}{}
\IfFileExists{microtype.sty}{% use microtype if available
  \usepackage[]{microtype}
  \UseMicrotypeSet[protrusion]{basicmath} % disable protrusion for tt fonts
}{}
\makeatletter
\@ifundefined{KOMAClassName}{% if non-KOMA class
  \IfFileExists{parskip.sty}{%
    \usepackage{parskip}
  }{% else
    \setlength{\parindent}{0pt}
    \setlength{\parskip}{6pt plus 2pt minus 1pt}}
}{% if KOMA class
  \KOMAoptions{parskip=half}}
\makeatother
\usepackage{xcolor}
\usepackage[margin=1in]{geometry}
\usepackage{color}
\usepackage{fancyvrb}
\newcommand{\VerbBar}{|}
\newcommand{\VERB}{\Verb[commandchars=\\\{\}]}
\DefineVerbatimEnvironment{Highlighting}{Verbatim}{commandchars=\\\{\}}
% Add ',fontsize=\small' for more characters per line
\usepackage{framed}
\definecolor{shadecolor}{RGB}{248,248,248}
\newenvironment{Shaded}{\begin{snugshade}}{\end{snugshade}}
\newcommand{\AlertTok}[1]{\textcolor[rgb]{0.94,0.16,0.16}{#1}}
\newcommand{\AnnotationTok}[1]{\textcolor[rgb]{0.56,0.35,0.01}{\textbf{\textit{#1}}}}
\newcommand{\AttributeTok}[1]{\textcolor[rgb]{0.13,0.29,0.53}{#1}}
\newcommand{\BaseNTok}[1]{\textcolor[rgb]{0.00,0.00,0.81}{#1}}
\newcommand{\BuiltInTok}[1]{#1}
\newcommand{\CharTok}[1]{\textcolor[rgb]{0.31,0.60,0.02}{#1}}
\newcommand{\CommentTok}[1]{\textcolor[rgb]{0.56,0.35,0.01}{\textit{#1}}}
\newcommand{\CommentVarTok}[1]{\textcolor[rgb]{0.56,0.35,0.01}{\textbf{\textit{#1}}}}
\newcommand{\ConstantTok}[1]{\textcolor[rgb]{0.56,0.35,0.01}{#1}}
\newcommand{\ControlFlowTok}[1]{\textcolor[rgb]{0.13,0.29,0.53}{\textbf{#1}}}
\newcommand{\DataTypeTok}[1]{\textcolor[rgb]{0.13,0.29,0.53}{#1}}
\newcommand{\DecValTok}[1]{\textcolor[rgb]{0.00,0.00,0.81}{#1}}
\newcommand{\DocumentationTok}[1]{\textcolor[rgb]{0.56,0.35,0.01}{\textbf{\textit{#1}}}}
\newcommand{\ErrorTok}[1]{\textcolor[rgb]{0.64,0.00,0.00}{\textbf{#1}}}
\newcommand{\ExtensionTok}[1]{#1}
\newcommand{\FloatTok}[1]{\textcolor[rgb]{0.00,0.00,0.81}{#1}}
\newcommand{\FunctionTok}[1]{\textcolor[rgb]{0.13,0.29,0.53}{\textbf{#1}}}
\newcommand{\ImportTok}[1]{#1}
\newcommand{\InformationTok}[1]{\textcolor[rgb]{0.56,0.35,0.01}{\textbf{\textit{#1}}}}
\newcommand{\KeywordTok}[1]{\textcolor[rgb]{0.13,0.29,0.53}{\textbf{#1}}}
\newcommand{\NormalTok}[1]{#1}
\newcommand{\OperatorTok}[1]{\textcolor[rgb]{0.81,0.36,0.00}{\textbf{#1}}}
\newcommand{\OtherTok}[1]{\textcolor[rgb]{0.56,0.35,0.01}{#1}}
\newcommand{\PreprocessorTok}[1]{\textcolor[rgb]{0.56,0.35,0.01}{\textit{#1}}}
\newcommand{\RegionMarkerTok}[1]{#1}
\newcommand{\SpecialCharTok}[1]{\textcolor[rgb]{0.81,0.36,0.00}{\textbf{#1}}}
\newcommand{\SpecialStringTok}[1]{\textcolor[rgb]{0.31,0.60,0.02}{#1}}
\newcommand{\StringTok}[1]{\textcolor[rgb]{0.31,0.60,0.02}{#1}}
\newcommand{\VariableTok}[1]{\textcolor[rgb]{0.00,0.00,0.00}{#1}}
\newcommand{\VerbatimStringTok}[1]{\textcolor[rgb]{0.31,0.60,0.02}{#1}}
\newcommand{\WarningTok}[1]{\textcolor[rgb]{0.56,0.35,0.01}{\textbf{\textit{#1}}}}
\usepackage{graphicx}
\makeatletter
\def\maxwidth{\ifdim\Gin@nat@width>\linewidth\linewidth\else\Gin@nat@width\fi}
\def\maxheight{\ifdim\Gin@nat@height>\textheight\textheight\else\Gin@nat@height\fi}
\makeatother
% Scale images if necessary, so that they will not overflow the page
% margins by default, and it is still possible to overwrite the defaults
% using explicit options in \includegraphics[width, height, ...]{}
\setkeys{Gin}{width=\maxwidth,height=\maxheight,keepaspectratio}
% Set default figure placement to htbp
\makeatletter
\def\fps@figure{htbp}
\makeatother
\setlength{\emergencystretch}{3em} % prevent overfull lines
\providecommand{\tightlist}{%
  \setlength{\itemsep}{0pt}\setlength{\parskip}{0pt}}
\setcounter{secnumdepth}{-\maxdimen} % remove section numbering
\ifLuaTeX
  \usepackage{selnolig}  % disable illegal ligatures
\fi
\usepackage{bookmark}
\IfFileExists{xurl.sty}{\usepackage{xurl}}{} % add URL line breaks if available
\urlstyle{same}
\hypersetup{
  pdftitle={R Notebook},
  hidelinks,
  pdfcreator={LaTeX via pandoc}}

\title{R Notebook}
\author{}
\date{\vspace{-2.5em}}

\begin{document}
\maketitle

\begin{Shaded}
\begin{Highlighting}[]
\FunctionTok{library}\NormalTok{(dplyr)}
\end{Highlighting}
\end{Shaded}

\begin{Shaded}
\begin{Highlighting}[]
\FunctionTok{library}\NormalTok{(psych)}
\FunctionTok{setwd}\NormalTok{(}\StringTok{\textquotesingle{}C:}\SpecialCharTok{\textbackslash{}\textbackslash{}}\StringTok{Users}\SpecialCharTok{\textbackslash{}\textbackslash{}}\StringTok{Chaitanya Kannan}\SpecialCharTok{\textbackslash{}\textbackslash{}}\StringTok{Desktop}\SpecialCharTok{\textbackslash{}\textbackslash{}}\StringTok{SDA Lab\textquotesingle{}}\NormalTok{)}
\end{Highlighting}
\end{Shaded}

Importing data which is in csv file

\begin{Shaded}
\begin{Highlighting}[]
\NormalTok{pdata }\OtherTok{=} \FunctionTok{read.csv}\NormalTok{(}\StringTok{\textquotesingle{}pizza\_delivery.csv\textquotesingle{}}\NormalTok{)}
\end{Highlighting}
\end{Shaded}

Printing Summary of data

\begin{Shaded}
\begin{Highlighting}[]
\FunctionTok{summary}\NormalTok{(pdata)}
\end{Highlighting}
\end{Shaded}

\begin{verbatim}
     day                date                time         operator            branch             driver         
 Length:1266        Length:1266        Min.   :12.27   Length:1266        Length:1266        Length:1266       
 Class :character   Class :character   1st Qu.:30.06   Class :character   Class :character   Class :character  
 Mode  :character   Mode  :character   Median :34.38   Mode  :character   Mode  :character   Mode  :character  
                                       Mean   :34.23                                                           
                                       3rd Qu.:38.58                                                           
                                       Max.   :53.10                                                           
  temperature         bill           pizzas         free_wine         got_wine      discount_customer
 Min.   :41.76   Min.   : 9.10   Min.   : 1.000   Min.   :0.0000   Min.   :0.0000   Min.   :0.000    
 1st Qu.:58.24   1st Qu.:35.50   1st Qu.: 2.000   1st Qu.:0.0000   1st Qu.:0.0000   1st Qu.:0.000    
 Median :62.93   Median :42.90   Median : 3.000   Median :0.0000   Median :0.0000   Median :0.000    
 Mean   :62.86   Mean   :42.76   Mean   : 3.013   Mean   :0.1809   Mean   :0.1485   Mean   :0.218    
 3rd Qu.:67.23   3rd Qu.:50.50   3rd Qu.: 4.000   3rd Qu.:0.0000   3rd Qu.:0.0000   3rd Qu.:0.000    
 Max.   :87.58   Max.   :75.00   Max.   :11.000   Max.   :1.0000   Max.   :1.0000   Max.   :1.000    
\end{verbatim}

Printing first few rows of the data

\begin{Shaded}
\begin{Highlighting}[]
\FunctionTok{head}\NormalTok{(pdata)}
\end{Highlighting}
\end{Shaded}

Number of rows in dataset:

\begin{Shaded}
\begin{Highlighting}[]
\FunctionTok{nrow}\NormalTok{(pdata)}
\end{Highlighting}
\end{Shaded}

Number of columns in dataset:

\begin{Shaded}
\begin{Highlighting}[]
\FunctionTok{ncol}\NormalTok{(pdata)}
\end{Highlighting}
\end{Shaded}

Count total rows in data frame with no NA values in any column of

\begin{Shaded}
\begin{Highlighting}[]
\FunctionTok{nrow}\NormalTok{(}\FunctionTok{na.omit}\NormalTok{(pdata))}
\end{Highlighting}
\end{Shaded}

\begin{verbatim}
[1] 1266
\end{verbatim}

Printing absolute frequencies of Branch attribute

\begin{Shaded}
\begin{Highlighting}[]
\FunctionTok{table}\NormalTok{(pdata}\SpecialCharTok{$}\NormalTok{branch)}
\end{Highlighting}
\end{Shaded}

Printing relative frequencies of Branch attribute length(pdata\$branch):
Gives the number of rows in the data

\begin{Shaded}
\begin{Highlighting}[]
\FunctionTok{table}\NormalTok{(pdata}\SpecialCharTok{$}\NormalTok{branch)}\SpecialCharTok{/}\FunctionTok{length}\NormalTok{(pdata}\SpecialCharTok{$}\NormalTok{branch)}
\end{Highlighting}
\end{Shaded}

Printing ECDF of branch

\begin{Shaded}
\begin{Highlighting}[]
\FunctionTok{plot}\NormalTok{(}\FunctionTok{ecdf}\NormalTok{(}\FunctionTok{table}\NormalTok{(pdata}\SpecialCharTok{$}\NormalTok{branch)), }\AttributeTok{xlab=}\StringTok{"Branch"}\NormalTok{, }\AttributeTok{ylab=}\StringTok{"ECDF"}\NormalTok{, }\AttributeTok{main=}\StringTok{"ECDF plot for Branch"}\NormalTok{, }\AttributeTok{col=}\FunctionTok{c}\NormalTok{(}\StringTok{\textquotesingle{}blue\textquotesingle{}}\NormalTok{,}\StringTok{\textquotesingle{}green\textquotesingle{}}\NormalTok{))}
\end{Highlighting}
\end{Shaded}

\begin{Shaded}
\begin{Highlighting}[]
\FunctionTok{plot}\NormalTok{(}\FunctionTok{ecdf}\NormalTok{(pdata}\SpecialCharTok{$}\NormalTok{temperature), }\AttributeTok{xlab=}\StringTok{"Temperature"}\NormalTok{, }\AttributeTok{ylab=}\StringTok{"ECDF"}\NormalTok{, }\AttributeTok{main=}\StringTok{"ECDF plot for Temperature"}\NormalTok{, }\AttributeTok{col=}\StringTok{"blue"}\NormalTok{)}
\end{Highlighting}
\end{Shaded}

\begin{Shaded}
\begin{Highlighting}[]
\FunctionTok{barplot}\NormalTok{(}\FunctionTok{table}\NormalTok{(pdata}\SpecialCharTok{$}\NormalTok{branch), }\AttributeTok{xlab=}\StringTok{"Branch"}\NormalTok{, }\AttributeTok{ylab=}\StringTok{"Count"}\NormalTok{, }\AttributeTok{main=}\StringTok{"Bar plot of Branch"}\NormalTok{, }\AttributeTok{col=}\FunctionTok{c}\NormalTok{(}\StringTok{\textquotesingle{}\#001199\textquotesingle{}}\NormalTok{,}\StringTok{\textquotesingle{}\#923\textquotesingle{}}\NormalTok{,}\StringTok{\textquotesingle{}green\textquotesingle{}}\NormalTok{))}
\end{Highlighting}
\end{Shaded}

\begin{Shaded}
\begin{Highlighting}[]
\FunctionTok{pie}\NormalTok{(}\FunctionTok{table}\NormalTok{(pdata}\SpecialCharTok{$}\NormalTok{branch), }\AttributeTok{labels=}\FunctionTok{names}\NormalTok{(}\FunctionTok{table}\NormalTok{(pdata}\SpecialCharTok{$}\NormalTok{branch)), }\AttributeTok{main=}\StringTok{"Bar plot of Branch"}\NormalTok{, }\AttributeTok{col=}\FunctionTok{c}\NormalTok{(}\StringTok{\textquotesingle{}\#001199\textquotesingle{}}\NormalTok{,}\StringTok{\textquotesingle{}red\textquotesingle{}}\NormalTok{,}\StringTok{\textquotesingle{}green\textquotesingle{}}\NormalTok{))}
\end{Highlighting}
\end{Shaded}

\begin{Shaded}
\begin{Highlighting}[]
\FunctionTok{hist}\NormalTok{(pdata}\SpecialCharTok{$}\NormalTok{time)}
\end{Highlighting}
\end{Shaded}

\begin{Shaded}
\begin{Highlighting}[]
\FunctionTok{hist}\NormalTok{(pdata}\SpecialCharTok{$}\NormalTok{time,}\AttributeTok{freq =} \ConstantTok{TRUE}\NormalTok{)}
\end{Highlighting}
\end{Shaded}

\begin{Shaded}
\begin{Highlighting}[]
\FunctionTok{plot}\NormalTok{(}\FunctionTok{density}\NormalTok{(pdata}\SpecialCharTok{$}\NormalTok{time, }\AttributeTok{kernel =} \StringTok{\textquotesingle{}gaussian\textquotesingle{}}\NormalTok{, }\AttributeTok{bw =} \FloatTok{0.5}\NormalTok{))}
\end{Highlighting}
\end{Shaded}


\end{document}
